% Do not forget to use:
% \usepackage[dvipsnames]{xcolor}
% \usepackage[breakable]{tcolorbox}
% \tcbuselibrary{theorems}

% COLORS
\newcommand{\lemcolor}{CarnationPink}
\newcommand{\thmcolor}{BrickRed}
\newcommand{\defcolor}{RoyalBlue}
\newcommand{\remcolor}{Salmon}
\newcommand{\propcolor}{Dandelion}
\newcommand{\corcolor}{RubineRed}

\newcommand{\excolor}{SpringGreen}
\newcommand{\expcolor}{Gray}
\newcommand{\conjcolor}{ForestGreen}

\newcommand{\cstcolor}{RoyalPurple}
\newcommand{\fncolor}{Orange}

\newcommand{\emphcolor}{WildStrawberry}
\newcommand{\cemph}[1]{\emph{\textcolor{\emphcolor!100}{#1}}}

\newcounter{main_counter}%
% THEOREMS ENV
\tcbset{
    genericstyle/.style={enhanced,breakable,theorem style=plain,colback=#1!5,colframe=#1!70!black,fonttitle=\bfseries,coltitle=#1!60!black,frame hidden,arc=0mm,left=0mm, right=0mm, top=0mm, bottom=0mm, boxsep=1mm, left skip=-1mm, right skip=-1mm}}
        %enhanced,breakable,theorem style=plain,colback=#1!5,colframe=#1!70!black,fonttitle=\bfseries,coltitle=#1!60!black,frame hidden,arc=0mm,boxsep=-1mm,
    % overlay first={
    %     \draw[line width=0.5pt, #1!70!black] (frame.north west)--(frame.north east);
    %     \draw[line width=0.2pt, #1!70!black, decorate, decoration={zigzag, segment length=3pt, amplitude=1pt}] (frame.south west)--(frame.south east);

    %     \draw[line width=0.5pt, #1!70!black] ($(frame.north west) + (0, 0.25pt)$)--(frame.south west);
    %     \draw[line width=0.5pt, #1!70!black] ($(frame.north east) + (0, 0.25pt)$)--(frame.south east);
    % },
    % overlay middle={%
    %     \draw[line width=0.1pt, #1!70!black, decorate, decoration={zigzag, segment length=3pt, amplitude=1.0pt}] (frame.south west)--(frame.south east);
    %     \draw[line width=0.1pt, #1!70!black, decorate, decoration={zigzag, segment length=3pt, amplitude=1.0pt}] (frame.north west)--(frame.north east);

    %     \draw[line width=0.5pt, #1!70!black] (frame.north west)--(frame.south west);
    %     \draw[line width=0.5pt, #1!70!black] (frame.north east)--(frame.south east);
    % },
    % overlay last={%
    %     \draw[line width=0.1pt, #1!70!black, decorate, decoration={zigzag, segment length=3pt, amplitude=1.0pt}] (frame.north west)--(frame.north east);
    %     \draw[line width=0.5pt, #1!70!black] (frame.south west)--(frame.south east);

    %     \draw[line width=0.5pt, #1!70!black] (frame.north west)--($(frame.south west) - (0, 0.25pt)$);
    %     \draw[line width=0.5pt, #1!70!black] (frame.north east)--($(frame.south east) - (0, 0.25pt)$);
    % },
    % overlay unbroken={%
    %     \draw[line width=0.5pt, #1!70!black] (frame.north west)--(frame.north east);
    %     \draw[line width=0.5pt, #1!70!black] (frame.south west)--(frame.south east);

    %     \draw[line width=0.5pt, #1!70!black] (frame.north east)--($(frame.south east) - (0, 0.25pt)$);
    %     \draw[line width=0.5pt, #1!70!black] (frame.north west)--($(frame.south west) - (0, 0.25pt)$);
    % },
    
    
\newtcbtheorem[use counter=main_counter]{clem}{Lemma}{genericstyle=\lemcolor}{lem}
\newtcbtheorem[use counter=main_counter]{cthm}{Theorem}{genericstyle=\thmcolor}{thm}
\newtcbtheorem[use counter=main_counter]{cdef}{Definition}{genericstyle=\defcolor}{def}
\newtcbtheorem[use counter=main_counter]{crem}{Remark}{genericstyle=\remcolor}{rem}
\newtcbtheorem[use counter=main_counter]{cprop}{Proposition}{genericstyle=\propcolor}{prop}
\newtcbtheorem[use counter=main_counter]{ccor}{Corollary}{genericstyle=\corcolor}{cor}
\newtcbtheorem[use counter=main_counter]{cconj}{Conjecture}{genericstyle=\conjcolor}{conj}
\newtcbtheorem[use counter=main_counter]{cexp}{Example}{genericstyle=\expcolor}{exp}

\newtcbtheorem{cex}{Exercise}{genericstyle=\excolor}{ex}
% PROOFS NAMES
\newcommand{\citedef}[1]{\normalfont{Definition \ref{def:#1}}}
\newcommand{\citelem}[1]{\normalfont{Lemma \ref{lem:#1}}}
\newcommand{\citethm}[1]{\normalfont{Theorem \ref{thm:#1}}}
\newcommand{\citerem}[1]{\normalfont{Remark \ref{rem:#1}}}
\newcommand{\citeprop}[1]{\normalfont{Proposition \ref{prop:#1}}}
\newcommand{\citecor}[1]{\normalfont{Corollary \ref{cor:#1}}}
\newcommand{\citecons}[1]{\normalfont{Construction \ref{cons:#1}}}

\newcommand{\ccitedef}[1]{\normalfont{\textbf{\textcolor{\defcolor!60!black}{Definition \ref{def:#1}}}}}
\newcommand{\ccitelem}[1]{\normalfont{\textbf{\textcolor{\lemcolor!60!black}{Lemma \ref{lem:#1}}}}}
\newcommand{\ccitethm}[1]{\normalfont{\textbf{\textcolor{\thmcolor!60!black}{Theorem \ref{thm:#1}}}}}
\newcommand{\cciterem}[1]{\normalfont{\textbf{\textcolor{\remcolor!60!black}{Remark \ref{rem:#1}}}}}
\newcommand{\cciteprop}[1]{\normalfont{\textbf{\textcolor{\propcolor!60!black}{Proposition \ref{prop:#1}}}}}
\newcommand{\ccitecor}[1]{\normalfont{\textbf{\textcolor{\corcolor!60!black}{Corollary \ref{cor:#1}}}}}
% \newcommand{\ccitecons}[1]{\normalfont{\textbf{\textcolor{\excolor!60!black}{Exercise \ref{cons:#1}}}}}



\newenvironment{lemproof}[1]{\begin{proof}{(\ccitelem{#1})}}{\end{proof}}
\newenvironment{thmproof}[1]{\begin{proof}{(\ccitethm{#1})}}{\end{proof}}
\newenvironment{propproof}[1]{\begin{proof}{(\cciteprop{#1})}}{\end{proof}}
\newenvironment{corproof}[1]{\begin{proof}{(\ccitecor{#1})}}{\end{proof}}
% \newenvironment{consproof}[1]{\begin{proof}{(\ccitecons{#1})}}{\end{proof}}



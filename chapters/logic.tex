\section{Logic}

\subsection{Connective}

Mathematics is a language made of sentences. In this section, we learn the basics of its grammar, and how to make well formed sentences. Then, given a well formed sentence, we see a general procedure to see wether a given sentence is true or false. We start with a few list of symbols, that constitute the basic alphabet of mathematics. It is primordial to know them, they are called \cemph{connectives}. We have

\begin{enumerate}
    \item The \cemph{and}, noted \( \land \). 
    \item The \cemph{or}, noted \( \lor \). 
    \item The \cemph{not}, noted \( \neg \). 
    \item The \cemph{implies}, noted \( \implies \). 
    \item The \cemph{equivalent}, noted \( \iff \). 
\end{enumerate}

As in any language, these symbols have a meaning, that somewhat correspond to the intuition. To understand it, let us abbreviate with the letter \( C \), the sentence \textit{the cat is orange}, and with \( D \) the sentence \textit{the dog has three legs}. Then, we write
\begin{equation*}
    C \land D
\end{equation*}
to mean that both the cat is eating \cemph{and} the dog is sleeping. Then whenever I see the cat, it is orange, and whenever I see the dog, it has three leg. 

Next is the or. It is a little bit different than what we are used to in the English language. We write
\begin{equation*}
    C \lor D
\end{equation*}
to mean that the cat is orange, \cemph{or} the dog has three leg. This means that at least one of the three following statements is true:
\begin{enumerate}
    \item The cat is orange,
    \item The dog has three legs,
    \item The cat is orange, and the dog has three legs.
\end{enumerate}
In math, the or connective need to to be exclusive. In \( C \lor D \), both \( C \) and \( D \) can be true. It means that after seeing the cat and the dog, I am guarantee that at least the cat will be orange, or the dog will have three legs, at least one of the two statements will be true. 

The next connective of interest is not. We write
\begin{equation*}
    \neg D
\end{equation*}
to mean that the dog has \cemph{not} three legs, that is whenever I will see the dog, it will have some number of legs and I am guarantee that this number is not three. It can be one, it can be four, it can be something else, but it will not be three.

\begin{cex}{}{and_not}
    Argue that \( \neg(C \land D) \) says the same thing as \( \neg C \lor \neg D \).
\end{cex}

We move on to equivalence. We write 
\begin{equation*}
    C \iff D
\end{equation*}
to mean that knowing that the cat is orange is the same thing as knowing that the dog has three leg. It means that if I go first see the cat to see it orange, then I do not need to go to the dog to see it has three leg. I already know it. Conversely, if I go first see the dog, and it has three legs, then I am sure that the cat is orange.

\begin{cex}{}{iff_sym}
    Argue that \( C \iff D \) says the same thing as \( D \iff C \).
\end{cex}

Finally, we have the implication. It is the most used of them all. We write
\begin{equation*}
    C \implies D
\end{equation*}
to mean that \cemph{if} the cat is orange, \cemph{then} the dog has three legs. It says that if I go see the cat and constate it is orange, then I am guarantee that the dog will have three legs. However, and this is very important, if I do not see that the cat is orange, then I \textit{cannot} say anything at all about the dog. 

\begin{cex}{}{iff_double_implies}
    Argue that \( (C \implies D) \land (D \implies C) \) says the same thing as \( C \iff D \). Is \( C \implies D \) saying the same thing as \(  D \implies C \)?
\end{cex}

Notice that when we will do math later, we will freely employ the symbol themselves, or their equivalent English terminology. In particular, we write "if \( X \) then \( Y \)" more often than "\( X \implies Y \)", but keep in mind that they are the same thing. We summarize these constructions with truth tables, you should refer to these when in doubt on what a sentence mean. Here is how to read it. The number \( 1 \) means True, the number \( 0 \) means False. In the table for \( \lor \), on a given row, a column gives a particular truth value to \( C \), to \( D \), and to the resulting \( C \lor D \). For instance, if \( C \) is true, \( D \) is false, we see that \( C \lor D \) is true.  
\begin{equation*}
    \begin{array}{| c | c |}
        \hline
        C & \neg C \\
        \hline
        0 & 1 \\
        \hline
        0 & 0 \\
        \hline
    \end{array},
    \begin{array}{| c | c | c |}
        \hline
        C & D & C \land D \\
        \hline
        0 & 0 & 0 \\
        \hline
        0 & 1 & 0 \\
        \hline
        1 & 0 & 0 \\
        \hline
        1 & 1 & 1 \\
        \hline
    \end{array},
    \begin{array}{| c | c | c |}
        \hline
        C & D & C \lor D \\
        \hline
        0 & 0 & 0 \\
        \hline
        0 & 1 & 1 \\
        \hline
        1 & 0 & 1 \\
        \hline
        1 & 1 & 1 \\
        \hline
    \end{array},
    \begin{array}{| c | c | c |}
        \hline
        C & D & C \iff D \\
        \hline
        0 & 0 & 1 \\
        \hline
        0 & 1 & 0 \\
        \hline
        1 & 0 & 0 \\
        \hline
        1 & 1 & 1 \\
        \hline
    \end{array},
    \begin{array}{| c | c | c |}
        \hline
        C & D & C \implies D \\
        \hline
        0 & 0 & 1 \\
        \hline
        0 & 1 & 1 \\
        \hline
        1 & 0 & 0 \\
        \hline
        1 & 1 & 1 \\
        \hline
    \end{array}
\end{equation*}


\subsection{Quantifiers}

So far, we cannot really say much. We need to introduce two new symbols:
\begin{equation*}
    \forall, \exists.
\end{equation*}
They will allow us to quantify, to say that all things in a big thing share the same property, or that there is some thing in a big thing that has a property. However, there is some subtleties that comes with those symbols, we need to use free variables. Earlier, I said \( D \) means that the dog has three legs. There is no room in this formula, everything is fixed. Allow me to do something, and replace three with the letter \( n \), that will I declare to be an unspecified natural number. Now, I write 
\begin{equation*}
    D(n)
\end{equation*}  
to mean that the dog has \( n \) legs, for some number \( n \), that I deliberately \textit{not} specify. This \( n \) is called a free variable, it can potentially be any natural number, and it is good to think of it as being \textit{all} the natural number at the same time. Now, if I take a natural number, say \( 7 \), then I will write \( D(7) \) to specify the unknown number \( n \) with 7, and \( D(7) \) means that the dog has seven leg. Notice that our previous sentence \( D \) is now the same thing as \( D(3) \). 

Is the sentence \( D(n) \) true or false? It does not make sense to ask this question. We cannot ask for the truth value of a sentence with free variables, we first need to specify a behavior for our free variable, and this is done with the quantifier. We write
\begin{equation*}
    \exists n, D(n)
\end{equation*}
to mean that there \cemph{exists} at least a value of \( n \) (like 4, 9, or seven billion) such that the dog has \( n \) legs. For instance, I know \( \exists n, D(n) \) is true because when I will look at my dog, I will count its number of legs, and see that there is \( n = 4 \). We say that \( 4 \) is a \cemph{witness} of \( \exists n, D(n) \). 

Next, we write
\begin{equation*}
    \forall n, D(n)
\end{equation*}
to mean that \cemph{for all} choice of number \( n \), my dog will have precisely this number of legs. Here, this is quite absurd, because my dog has one and only one number of leg. But consider the following:
\begin{equation*}
    \forall n, (5 \leq n \implies \neg D(n)).
\end{equation*} 
It means that for all number \( n \), if the number \( n \) is greater or equal to 5, then my dog has not \( n \) legs. This feels more true, as I know indeed that my dog has four legs. 

\begin{cex}{}{write_sentences}
    Is \( \exists n, \neg D(n) \) true? Can you rewrite \( \neg \exists n, \neg D(n) \) with something with less symbols?
\end{cex}


\subsection{Practical sentences, and how to do proofs}

This was only the tip of the iceberg, and logic is a very powerful language to talk with math. Ultimately, we will also want to do proofs. This section is a practical place that you are invited to reread every time you do not know how to start a proof, or do an exercise. 

\cemph{TODO}
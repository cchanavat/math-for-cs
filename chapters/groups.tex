\section{Groups}

Groups are mathematical structures that arise everywhere. Groups encode symmetries in structures, and symmetries are prevalent in math.  As this class is computer-science oriented, we will first introduce monoids, that are object slightly more general than groups, and that anyone in computer science already encounter. Typically, when we consider the regular expression \( (ab)^* \), we are considering the free monoid on the alphabet \( \{ a, b \} \). 

\subsection{Monoids}

\begin{cdef}{}{binary_op}
    Let \( X \) be a set. A \cemph{binary operation} \( \cdot \) on \( X \) is a function \( \cdot : X \times X \to X \). Instead of writing \( \cdot(x, y) \) for the application of \( \cdot \) to \( (x, y) \), we typically write \( x \cdot y \).
\end{cdef}

\begin{cexp}{}{example_bin_op}
    This definition should not be new for you, it is just the abstract version of things we already know. 
    \begin{itemize}
        \item The function \( + : \bb N \times \bb N \to \bb N \) is a binary operation.
        \item The function \( \times : \bb R \times \bb R \to \bb R \) is a binary operation.
        \item etc.
    \end{itemize}
\end{cexp}

\begin{cdef}{}{monoid}
    Let \( (X, \cdot) \) be a set with a binary operation. We say that \( (X, \cdot) \) is a monoid if
    \begin{enumerate}
        \item There exists a particular element \( e \in X \), called the \cemph{neutral element}, such that:
        \begin{equation*}
            \forall x \in X, e \cdot x = x = x \cdot e.
        \end{equation*}
        \item The binary operation is \cemph{associative}, that is:
        \begin{equation*}
            \forall x, y, z \in X, x\cdot(y \cdot z) = (x \cdot y) \cdot z.
        \end{equation*}
            In that case, we are allowed to write \( x \cdot y \cdot z  \) to mean either \( x\cdot(y \cdot z) \) or \( (x \cdot y) \cdot z \), as they are equal.
    \end{enumerate}
\end{cdef}

\begin{cexp}{}{monoid_R_N}
    Prove that \( (\bb N, +) \) is a monoid. Is \( (\bb R, \times) \) a monoid? Prove that \( (\bb R^*, \times) \) is a monoid, where by \( \bb R^* \), we mean the set of real number with \( 0 \) removed. What is its neutral element?
\end{cexp}

\begin{cexp}{}{trivial_monoid}
    A very important monoid is the set with one element \( \{ e \} \). The binary law is (necessarily) defined by \( e \cdot e = e \), and the neutral element is (necessarily) \( e \). Check that this is indeed a monoid.
\end{cexp}

\begin{clem}{}{unique_neutral}
    Neutral elements are unique, that is in a monoid \( (X, \cdot, e) \), if there is an element \( e' \in X \) such that 
    \begin{equation*}
        \forall x \in X, e' \cdot x = x = x \cdot e',
    \end{equation*}
    we have \( e = e' \).
\end{clem}
\begin{lemproof}{unique_neutral}
    Suppose \( e' \) is another neutral element for all \( x \) we have 
    \begin{equation*}
        e' \cdot x = x, 
    \end{equation*}
    so in particular letting \( x = e \), we get \( e' \cdot e = e \). Now, \( e \) is also neutral element, so for all \( x \), we have \( x \cdot e = x \), hence with \( x = e' \), we get \( e' = e' \cdot e \), so \( e' = e \).
\end{lemproof}

\begin{cdef}{}{commutative_monoid}
    Let \( (X, \cdot, e) \) be a monoid with binary operation \( \cdot \) and neutral element \( e \). We say that \( X \) is \cemph{commutative} if 
    \begin{equation*}
        \forall x,y \in X, x \cdot y = y \cdot x.
    \end{equation*} 
\end{cdef}

\begin{crem}{}{notation_monoid}
    Here are some common abuse of notation that we do in group theory. We say that \textit{\( X \) is a monoid}, where we are supposed to say \( (X, \cdot, e) \) is a monoid, the data of the binary law and the neutral element being part of the definition. As they are often explicit from the context, we tend to avoid it, and say simply that \( X \) is a monoid.

    More often than not, when the monoid is commutative, we write its law \( + \), and \( 0 \) its neutral element. Beware that these \( + \) and \( 0 \) have a priori nothing to do with the \( + \) and \( 0 \) of the natural numbers. It is just that this notation helps us remember that the monoid is commutative, as is the monoid \( (\bb N, +, 0) \).
    
    Also, when \( (X, \cdot, e) \) is a monoid, we also like to write \( xy \) for \( x \cdot y \), like we often write \( st \) for \( s \times t \).    
\end{crem}

\begin{cdef}{}{morphism_monoid}
    Let \( (X, \cdot, e_X), (Y, \cdot, e_Y) \) be monoids, a function \( f : X \to Y \) is a \cemph{morphism of monoids} if
    \begin{equation*}
        \forall x, y, f(x \cdot y) = f(x) \cdot f(y),
    \end{equation*}
    and 
    \begin{equation*}
        f(e_X) = e_Y,
    \end{equation*}
    that is \( f \) preserves the monoid laws, and it sends the neutral element to the neutral element. 
\end{cdef}

\begin{cex}{}{exaple_mph_monoid}
    Prove that the exponential function \( \exp : (\bb R, +) \to (\bb R^*, \times) \) is a morphism of monoids. Is the function 
    \fun{f}{(\bb N, +, 0)}{(\bb N, +, 0)}{x}{x + 1}
    a morphism of monoids?
\end{cex}

\begin{cdef}{}{sumonoid}
    Let \( X \) be a monoid, and let \( A \subseteq X \). We say that \( A \) is a \cemph{submonoid} if it contains the neutral element and is closed under the monoid law, that is \( e \in A \), and for all \( x, y \in A \), \( xy \in A \).
\end{cdef}

\begin{cdef}{}{kernel_image}
    Let \( f : X \to Y \) be a morphism of monoids. We define the \cemph{kernel} of \( f \) to be the set
    \begin{equation*}
        \ker(f) \defeq f^{-1}(\{ e_Y \}) = \{ x \in X \mid f(x) = e_Y \} \subseteq X
    \end{equation*}
    and the \cemph{image} of \( f \) to be the set
    \begin{equation*}
        \im(f) \defeq \{ f(x) \mid x \in X \} \subseteq Y.
    \end{equation*}
    (which is the same thing as the set-theoretical image of Definition \ref{def:image_preimage}).
\end{cdef}

The kernel and the image have in fact a structure of monoid, so when one is given with a morphism of monoid, one get two submonoids for free.

\begin{clem}{}{submonoid_kernel_image}
    Let \( f : X \to Y  \) be a morphism of monoid. Then \( \ker(f) \) is a submonoid of \( X \), and \( \im(f) \) is a submonoid of \( Y \).
\end{clem}
\begin{lemproof}{submonoid_kernel_image}
    By definition of a morphism of monoid, \( f(e_X) = e_Y \), this means \( e_X \in \ker(f) \). If \( x, y \in \ker(f) \), then 
    \begin{equation*}
        f(xy) = f(x)f(y) = e_Ye_Y = e_Y,
    \end{equation*}
    so \( xy \in \ker(f) \). This proves \( \ker(f) \) is a submonoid of \( X \).

    Now for the image, again \( f(e_X) = e_Y \), so \( e_Y \in f(X) = \im(f) \). If \( y, y' \in \im(f) \), then by definition there exist \( x, x' \in X \) such that \( f(x) = y \) and \( f(x') = y' \), so \( yy' = f(x)f(x') = f(xx') \), meaning that \( yy' \in \im(f) \). This proves that \( \im(f) \) is a submonoid of \( Y \). 
\end{lemproof}

\begin{cdef}{}{product}
    Let \( X, Y \) be monoids. We define the product of \( X, Y \) to be the monoid whose underlying set is \( X \times Y \), the neutral element is the couple \( (e_X, e_Y) \), and the law is defined pointwise, that is 
    \begin{equation*}
        (x, y) \cdot (x', y') \defeq (x\cdot x', y \cdot y').
    \end{equation*}
    More generally, if \( (X_i)_{i\in I} \) is a family of monoids, we define the product \( \prod_{i \in I} X_i \) to be the monoid whose underlying set is \( \prod_{i \in I} X_i \), whose neutral element is \( (e_{X_i})_{i \in I} \), and whose law is defined pointwise with
    \begin{equation*}
        (x_i)_{i \in I} \cdot (x'_i)_{i \in Y} \defeq (x_i \cdot x'_i)_{i \in I}.
    \end{equation*}
\end{cdef}

\begin{cex}{}{product_is_monoid}
    Prove that if \( (X_i)_{i\in I} \) is a family of monoids, indeed \( \prod_{i \in I} X_i \) is a monoid.
\end{cex}

Given a set \( X \), how can we make it a monoid \( X^* \) such that elements of \( X \) are inside \( X^* \)? To see that, first suppose \( X = \{ x \} \), a set with an element. We construct a monoid \( X^* \). As it is a monoid, it must have a neutral element, we call it \( e \). We also want \( x \in X^* \), so we put it there. So far our monoid \( X^* \) has elements \( e \) and \( x \). But now, we can also consider \( x\cdot x \), a priori, this element does not belong to \( X^* \), but it should still exists, so we add it, and we call it \( xx \) for simplicity. Now our monoid has elements \( \{ e, x, xx \} \), and again we can consider \( x \cdot (x\cdot x) \), or \( (x \cdot x) \cdot x \). Those elements will have to be the same, so we add another element \( xxx \) to the monoid. And we continue forever. The end result will be that the elements of \( X^* \) are strings of \( x \)'s, the monoid operation is concatenation, and the neutral element is the empty string. This indeed satisfies the axioms of monoid, as concatenating is associative, and concatenating the empty string to the left or the right of a word does not change it. Let us give a more general definition, when \( X \) is any set.

\begin{cdef}{}{free_monoid}
    Let \( X \) be a set. We define \( X^* \) to be the monoid whose elements are finite strings \( x_1 \dots x_n \) with \( x_i \in X \), whose law is concatenation, and whose neutral element is concatenation. This indeed defines a monoid, as concatenation is associative, and concatenating with the empty string does not change a string.
\end{cdef}

\begin{clem}{}{map_free_monoid}
    Let \( X \) be any set, and \( (M, \cdot, e_M) \) be a monoid. Then any set-theoretical function \( f : X \to M \) gives rise to a morphism of monoid \( f^* : X^* \to M \) by letting
    \begin{equation*}
        f^*(x_1 x_2 \dots x_n) = f(x_1)\cdot f(x_2) \dots f(x_n),
    \end{equation*}
    where we allow \( n = 0 \), and we mean \( f^*(e) = e_M \).
\end{clem}
\begin{lemproof}{map_free_monoid}
    By definition, the map indeed maps the neutral element to the neutral element. If \( x_1\dots x_n, y_1 \dots y_m \in X^* \), then 
    \begin{equation*}
        f^*(x_1\dots x_n y_1 \dots y_m) = f(x_1) \dots f(x_n)f(y_1) \dots f(x_m) = f^*(x_1\dots x_n)f^*(y_1\dots y_m).
    \end{equation*}
\end{lemproof}

\subsection{Groups}

Monoids are interesting objects, but if we ask moreover that every element has an inverse, a whole new world appears, it is the one of groups. 
\begin{cdef}{}{group}
    A \cemph{group} \( (G, \cdot, e) \) is a monoid together with a function \( \inv - : G \to G \) that sends \( x \in G \) to \( \inv x \), called the \cemph{inverse} of \( x \), and such that
    \begin{equation*}
        \forall x \in G, x \cdot \inv x = e = \inv x \cdot x.
    \end{equation*}
    A group is \cemph{abelian} if its underlying monoid is abelian, see Definition \ref{def:commutative_monoid}.
\end{cdef}


\begin{crem}{}{notation_group}
    The Remark \ref{rem:notation_monoid} also applies for groups, for instance we will often write \( + \) for the law of an abelian group. Furthermore, we extend this notation to \( - x \) to mean \( \inv x \) in the case where the group is abelian.
\end{crem}

\begin{cexp}{}{example_groups}
    \begin{itemize}
        \item \( (\bb Z, +, 0, -) \) is an abelian group.
        \item \( (\bb R^*, \times, 1, x \mapsto 1/x) \) is an abelian group.
        \item Let \( X \) be a set, call \( \mathrm{Bij}(X) \) the set of all bijective function \( f : X \to X \). This set is a (non-abelian) group with composition. What is the inverse of a function? What is the neutral element? 
    \end{itemize}    
\end{cexp}

\begin{crem}{}{monoid_applies_group}
    What we said previously about monoid, can also be extended to groups. In particular, neutral element is unique.
\end{crem}

\begin{clem}{}{inverse_unique}
    Let \( G \) be a group, and let \( x \in G \). Suppose we have \( x' \in G \) such that \( x' \cdot x = e \) or \( x \cdot x' = e \), then \( x' = \inv x \).
\end{clem}
\begin{lemproof}{inverse_unique}
    Suppose for instance \( x' \cdot x = e \), then multiplying by \( \inv x \) both sides yields
    \begin{equation*}
        x' \cdot x \cdot \inv x = e \cdot \inv x,
    \end{equation*}
    which simplifies to \( x' = \inv x \). 
\end{lemproof}